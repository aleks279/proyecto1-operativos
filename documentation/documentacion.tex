\documentclass{article}
\usepackage{graphicx}
\usepackage{fancyhdr}
\usepackage{listings}

\let\<\textless
\let\>\textgreater

\graphicspath{ {images/} }
\pagestyle{fancy}
\fancyhf{}
\rhead{Proyecto \#1}
\rfoot{P\'agina \thepage}

\begin{document}
\begin{titlepage}
  \centering
  {\scshape\LARGE Instituto Tecnol\'ogico de Costa Rica \par}
  \vspace{1cm}
  {\scshape\Large Proyecto \#1\par}
  \vspace{1.5cm}
  {\Large\itshape Luis Castillo\par}
  {\Large\itshape Janis Cervantes\par}
  {\Large\itshape Sa\'ul Zamora\par}
  \vfill
  profesor\par
  Kevin Moraga \textsc{}

  \vfill

% Bottom of the page
  {\large \today\par}
\end{titlepage}

\section{Introducci\'on}
Se debe realizar una re\-implementaci\'on de algunas de las funciones de la biblioteca pthreads de C del sistema operativo GNU\/Linux.

Como rubro extra, se comprob\'o el funcionamiento de la nueva librer\'ia my\_pthreads.h reemplazando pthreads en la tarea \#2, Net Neutrality.

\section{Ambiente de desarrollo}
\begin{itemize}
  \item M\'aquina virtual: VMware Workstation 14 Pro
  \item Sistema operativo utilizado: Linux Ubuntu 17.10 LTS
  \item gcc (Ubuntu 7.2.0-8ubuntu3.2) 7.2.0
\end{itemize}

\section{Estructuras de datos usadas y funciones}
\subsection{MyPthreads}
Se realiz\'o una reimplementaci\'on de la bilbioteca de pthreads, con las siguientes funciones:

\begin{itemize}
  \item my\_thread\_create
  \item my\_thread\_create\_with\_params
  \item my\_thread\_end
  \item my\_thread\_yield
  \item my\_thread\_join
  \item my\_thread\_detach
  \item my\_mutex\_init
  \item my\_mutex\_destroy
  \item my\_mutex\_lock
  \item my\_mutex\_unlock
\end{itemize}

\section{Instrucciones de ejecuci\'on}
Para compilar el servidor prethread (y el cliente) de la tarea \#2 y comprobar su funcionamiento con la nueva librer\'ia my\_pthreads.h, se siguen los siguientes pasos:
\begin{itemize}
  \item Servidor pre-Thread:
  \begin{itemize}
    \item \emph{gcc prethread-Server.c -o prethread-Server}
  \end{itemize}
  \item Para el cliente:
  \begin{itemize}
    \item \emph{gcc client.c -o client}
  \end{itemize}
\end{itemize}

Para la ejecuci\'on del servidor y cliente se sigue el siguiente proceso:
\begin{itemize}
  \item Servidor pre-Thread:
  \begin{itemize}
    \item \emph{./prethread-Server -n \<num-hilos\> -P \<prioridad\> -r \<path-a-recursos\> -p \<puerto\>}
  \end{itemize}
  \item Para el cliente:
  \begin{itemize}
    \item \emph{./client -h \<host\> -p \<puerto\>}
  \end{itemize}
\end{itemize}

\section{Bit\'acora de trabajo}
\subsection{Sa\'ul Zamora}
\begin{itemize}
  \item 09-04-2018:
  \begin{itemize}
    \item 2 horas - Investigar implementacion de threads usando context.
  \end{itemize}
  \item 10-04-2018:
  \begin{itemize}
    \item 2 horas - Implementaci\'on de librer\'ia de hash\_table.
  \end{itemize}
  \item 11-04-2018:
  \begin{itemize}
    \item 2 horas - Implementaci\'on de librer\'ia de linked\_list.
  \end{itemize}
  \item 12-04-2018:
  \begin{itemize}
    \item 2 horas - Implementaci\'on de librer\'ia my\_pthreads.
  \end{itemize}
  \item 13-04-2018:
  \begin{itemize}
    \item 2 horas - Implementaci\'on de librer\'ia my\_pthreads.
  \end{itemize}
  \item 14-04-2018:
  \begin{itemize}
    \item 4 horas - Implementaci\'on de librer\'ia my\_pthreads.
  \end{itemize}
  \item 14-04-2018:
  \begin{itemize}
    \item 2 horas - Modificaci\'on en la tarea de net neutrality para comprobar el funcionamiento de la nueva librer\'ia my\_pthreads.
  \end{itemize}
\end{itemize}
Total de horas trabajadas: 16 horas.

\section{Comentarios finales}
\begin{itemize}
  \item 
\end{itemize}

\section{Conclusiones}
\begin{itemize}
  \item 
\end{itemize}

\begin{thebibliography}{99}
  \bibitem{context} Anon, (n.d.). Context switching - ucontext\_t and makecontext(). [online] Available at: \texttt{https://stackoverflow.com/questions/21468529/context-switching-ucontext-t-and-makecontext.}
\end{thebibliography}
\end{document}